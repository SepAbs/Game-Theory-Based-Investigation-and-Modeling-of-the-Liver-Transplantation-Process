\documentclass[12pt]{article}
\usepackage{xepersian}
\usepackage{graphicx}
\usepackage{caption}
\usepackage{amsmath, amsfonts, amssymb, nccmath}
\usepackage{multicol, multirow}
\settextfont{B Yas}
\title{بررسی و مدلسازی روند پیوند کبد با استفاده از نظریه بازی ها}
\author{سپهر عباسپور، حسین نظری و سینا نظری}
\begin{document}
\maketitle
\begin{center}
\Large
 \textbf{چکیده}
\end{center}
\\
\normalsize
سالانه میلیون‌ها نفر در صف دریافت کبد جهت انجام عمل پیوند کلیه هستند؛ بنابراین بایستی سیاستی اتّخاذ شود تا بیماران دارای صلاحیّت بیشتر، زودتر به کبد مورد نیاز دست پیدا کنند.
امّا خاستگاه این سیاست‌گذاری چیست؟
سیاست‌گذاری هدف به کمک علم نظریه بازی صورت خواهد گرفت؛ بدین صورت که پزشک معالج و بیماران نیازمند دریافت کبد بصورت بازیکنانی در نظر گرفته شده و فعّالیّت‌های سودمند هر یک از آنان تحلیل می‌شوند.
تحت این فرآیند می‌توان به بیماران به عدالت کبد رسانید و آنان را نجات داد.
\\
\section{مقدمه}
\\

هدف این ارائه بررسی رفتار های اجتماعی پیرامون عمل
پیوند کبد و بهینگی آن به وسیله نظریه بازی هاست.
پیوند کبد بعد از پیوند کلیه مرسوم ترین نوع پیوند عضو در دنیاست به
شکلی که تنها در سال 2015 در آمریکا بیش از 7100 پیوند کبد
صورت گرفت که 600 نفر از آنها کودک بودند.

پیوند کبد به نسبت عمل کم خطری هم می باشد. بنا بر تحقیقات صورت
گرفته 83 درصد عمل شوندگان پس از یک سال زنده می‌مانند و 72 درصد پس
از سه سال و حدود 53 درصد تا بیست سال بعد از عمل زنده می‌مانند.

با اینکه در سال های قبل از 2015 بیشترین دلیل نیاز به عمل پیوند کبد
در آمریکا هپاتیت \lr{c} بوده ولی تا سال 2019 جای خود را به مصرف
بی رویه الکل داده است با 33.1 درصد فراوانی در مردان و 21.1 درصد در
زنان.

با افزایش تعداد کسانی که به اهدای کبد نیاز دارند و کاهش اهدا کنندگان،
تعداد کسانی که در لیست انتظار جان خود را از دست می‌دهند در دهه
گذشته افزایش 30 درصدی داشته.
با این تفاسیر اینکه چه کسی در لیست انتظار قرار می‌گیرد یک مسئله اخلاقی بسیار مهم است.
\lr{Diamond} 
در سال 1986 در مقاله خودش با عنوان \lr{Application of game theory to
medical decision-making} جزو اولین کسانی بود به به مسئله کبد پرداخت. در مقاله ذکر
شده دایموند برای مسئله بیماری هپاتیت کبدی و اینکه آیا دکتر بهتر است درمان با استروید را
شروع کند یا عمل خطرناکی را انجام دهد از نظریه بازی ها استفاده کرد. دکتر می‌تواند هر کدام
از دو تشخیص را بدهد و بیمار آزاد است که انتخاب کند درمان را بپذیرد یا نه. این مسئله شباهت
زیادی به مسئله پیش رو دارد.
\\

در سال 2015 \lr{Djulbegovic}\ در مقاله \lr{Modern health care as a game theory
	problem} این مساله را با مشابهت به بازی زندانی حل کرد. نویسنده به شکل خلاقانه ای
احساسات را هم در بازی دخیل کرده بود تا دقیق تر موضوع را بررسی کند. ما 
در ادامه از این
ایده الهام گرفته و با استفاده از نظریه بازی‌ها یک مدل اجتماعی برای این مهم
ارائه داده و برای سادگی بیشتر تنها بیماران مبتلا به \lr{ALD} \LTRfootnote{\lr{Alcoholic Associated Liver Disease}} را بررسی می‌کنیم.
\\
\section{مقدمات مدل سازی}

بکارگیری نظریه بازی برای یافتن این پاسخ فلسفی نیز می باشد که آیا اخلاقیات دلیل بر احساس گناه است یا برعکس.

بازی دارای دو بازیکن است: دکتر\LTRfootnote{\lr{Doctor}} و بیمار\LTRfootnote{\lr{Patient}}
هر کدام از بازیکن ها دارای لیست اعمال زیر است:
\\
\\
\textbf{پزشک:}

1. همکاری\LTRfootnote{\lr{Cooperation}}(\lr{C}): در این حالت دکتر اسم بیمار را به لیست انتظار اضافه می‌کند.

2. عدم همکاری موقت\LTRfootnote{\lr{No Cooperation T}}(\lr{NCt}): در این حالت دکتر اسم بیمار را اضافه نمی‌کند اما اگر بیمار نظرش را تغییر دهد شاید اسم او را اضافه کند

3. عدم همکاری قطعی\LTRfootnote{\lr{No Cooperation D}}(\lr{NCd}): در این حالت دکتر اسم بیمار را تحت هیچ شرایطی به لیست انتظار اضافه نمی‌کند.
\\
\\
\textbf{بیمار:}

1. همکاری (\lr{C}): در این حالت بیمار به توصیه های دکتر گوش می‌دهد تا عمل به خوبی انجام شود و پس از عمل بتواند به زندگی سالم خود ادامه دهد.

2. عدم همکاری (\lr{NC})\LTRfootnote{\lr{No Cooperation}}: در این حالت بیمار به توصیه های دکتر گوش نمی‌دهد و برای مثال با مصرف دوباره الکل جان خود را به خطر می‌اندازد و امکان زندگی سالم را از بیماری که حاضر به تغییر است می‌گیرد.
\\
\section{تابع سود و ارجحیت های بازیکنان}

\textbf{پزشک:}

تابع سود پزشک را با \lr{$D_{\textit{ij}}$} نمایش می‌دهیم، که \lr{\textit{i}} نماد تصمیم دکتر و \lr{\textit{j}} نماد تصمیم بیمار است. برای مثال $D_{31}$ نماد سود دکتر است وقتی که خودش تصمیم گرفته تحت هیچ شرایطی اسم بیمار را اضافه نکند ولی بیمار تصمیم گرفته همکاری کند.\\

\textbf{بیمار:}

تابع سود بیمار را مشابه پزشک با \lr{$P_{\textit{ij}}$}  نشان می‌دهیم، که مشابهاً \lr{\textit{i}} نماد تصمیم پزشک و \lr{\textit{j}} نماد تصمیم بیمار است.

بنابراین جدول بازی به صورت زیر در خواهد آمد:

\begin{table}[h!]
\caption{تابع سود بازیکنان}
\begin{center}
\begin{tabular}{|c|c|c|c|c|}
\hline
& \multicolumn{4}{|c|}{پزشک}& \cline{1-5}
بیمار & & \lr{C} & \lr{NCt} & \lr{NCd}\\
\cline{2-5}
& \lr{C} & \lr{$P_{11}$, $D_{11}$} & \lr{$P_{21}$, $D_{21}$} & \lr{$P_{31}$, $D_{31}$}\\
\cline{2-5}
& \lr{NC} & \lr{$P_{12}$, $D_{12}$} & \lr{$P_{22}$, $D_{22}$} & \lr{$P_{32}$, $D_{32}$}
\cline{2-6}
\end{tabular}
\end{center}
\end{table}
همچنین ارجحیت های بازیکنان نیز به شرح زیر است:\\

\textbf{پزشک:}

به وضوح میتوان دید که انتظار داریم نامساوی زیر برقرار باشد.
\[
\overbrace{D_{11}}^{\text{\large مثبت سود}} > \underbrace{D_{22} > D_{32}}_{\text{\large منفی سود احتمال}} > \overbrace{D_{21} > D_{31} > D_{12}}^{\text{\large منفی سود}}
\]

\textbf{بیمار:}

برای بیمار نیز انتظار داریم که نامساوی زیر برقرار باشد.
\[
\overbrace{P_{11}}^{\text{\large مثبت سود}} > \underbrace{P_{12}}_{\text{\large منفی سود احتمال}} >\overbrace{P_{21}> P_{22} > P_{31} > P_{32}}^{\text{\large منفی سود}}
\]

به وضوح دادن ارجحیت به این شکل برای تحلیل مسئله به ما کمکی نمی‌کند. بنابراین باید پارامتر هر تصمیم را مشخص کنیم.

\subsection{سود خالص}
سود خالص هر بازیکن، سود اولیه او از وضعیت بیمار در لیست است.
پس سود خالص پزشک را با $ U_{\textit{\lr{i}}}$ و سود خالص بیمار را با $ V_{\textit{\lr{i}}}$ نمایش می‌دهیم که \lr{\textit{i}} نماد تصمیم پزشک است. برای مثال سود خالص بیمار اگر موقتا از لیست حذف شود برابر با $V_2 $ است.
به وضوح داریم که
\[
U_1>U_2>U_3
\]
\[
V_1>V_2>V_3
\]

\subsection{احساسات}

برای تحلیل بهتر احساسات هر بازیکن را به عنوان یک پارامتر مثبت ثابت در نظر می‌گیریم.\\
\\
1. \textbf{گناه(\lr{G}) \LTRfootnote{\lr{Guilt}}}

احساس گناه وقتی ناشی می‌شود که دکتر کبد سالم را به بیماری بدهد که به سلامتی‌اش توجهی نمی‌کند.\\
\\
2. \textbf{استیصال(\lr{F}) \LTRfootnote{\lr{Frustration}}}

احساس استیصال وقتی ناشی می‌شود که بیمار به سلامتی خود اهمیت نمی‌دهد و کبد هم دریافت نمی‌کند به همین واسطه مرگ قریب الوقوعی دارد.\\
\\
3. \textbf{پشیمانی(\lr{R}) \LTRfootnote{\lr{Regret}}}

احساس پشیمانی در صورتی حاصل می‌شود که دکتر کبد را به بیماری که به سلامتی خود اهمیت می‌دهد، ندهد.
\\

برای حالات مختلف پشیمانی و استیصال از $F_{\textit{\lr{k}}}^{\textit{\lr{l}}}$ استفاده می‌کنیم که \lr{\textit{F}} نماد احساس، \lr{\textit{l}} نماد قطعیت تصمیم پزشک و \lr{\textit{k}} نماد بازیکن است.
برای مثال، $F_{\textit{\lr{D}}}^{\textit{\lr{d}}}$ احساس استیصال پزشک \lr{D} است وقتی که به طور قطعی  \lr{d} بیمار را از لیست حذف کند.
همچنین به طور پیش فرض داریم که
\[
R^d - F^d > R^t - F^t ,
\]
\[
G > R^d > R^t > F^d > F^t 
\]

دو پارامتر دیگر هم در بازی برای بیمار در نظر میگیریم یکی آسیب(\lr{Harm}) حاصل از نوشیدن الکل است که با \lr{H} نشان می‌دهیم. همچنین سود(\lr{Benefit}) حاصل از ننوشیدن الکل بر سلامت بیمار است که آن را با \lr{B} نمایش می‌دهیم.
به طور فرض داریم که آسیب نوشیدن الکل از سلامت ترک الکل اهمیتی بیشتری دارد چون ممکن است به مرگ بیمار منجر شود.
\[
|H| > |B|
\]

تا به اینجای کار تمام پارامتر های ما ثابت های عددی تاثیر گذار بر روابط اجتماعی بودند.

دو متغیر زیر را برای تحلیل حساسیت سنجی موضوع در نظر میگیریم

$ :\gamma$
نماد لذت حاصل از نوشیدن الکل است این پارامتر متغیر است و در تحلیل های ما کم یا زیاد می‌شود.\\

$ :\beta$
برای دکتر نماد این موضوع است که چقدر برایش اهمیت دارد که کبد را برای فردی که به سلامتی خود اهمیت می‌دهد حفظ کند. اینکه کبد چقدر مهم تر از بیمار است.\\

حال تابع سود بازیکنان را بر اساس پارامتر های توصیف شده بازنویسی می‌کنیم.

\begin{table}[h!]
\caption{تابع سود بازیکنان بر اساس پارامترهای توصیف شده}
\begin{center}
\begin{tabular}{|c|c|}
\hline
& تابع سود &
\hline
پزشک &
\begin{aligned}
\\
&D_{11} = V_1\\
&D_{12} = V_1 - G \cdot (V_1 − V_2)\\
&D_{21} = V_2 - R_D^t \cdot (V_1 - V_2)\\ 
&D_{22} = V_2 - F_D^t \cdot (V_1 - V_2)\\
&D_{31} = V_3 - R_D^d \cdot (V_1 - V_3) + \beta\\
&D_{32} = V_3 - F_D^d \cdot (V_1 - V_3) + \beta\\
\\
\end{aligned}
\\
\hline
بیمار &
\begin{aligned}
\\
&P_{11} = U_1 + B\\
&P_{12} = U_1 - H + \gamma\\
&P_{21} = U_2 - F_P^t \cdot (U_1 - U_2) + B\\
&P_{22} = U_2 - R_P^t \cdot (U_1 - U_2) - H + \gamma\\
&P_{31} = U_3 - F_P^d \cdot (U_1 - U_3) + B\\
&P_{32} = U_3 - R_P^d \cdot (U_1 - U_3) - H + \gamma\\
\\
\end{aligned}
\\
\hline
\end{tabular}
\end{center}
\end{table}

\section{تعادل های نش}
\subsection{تعادل های نش محض}

در ادامه سعی داریم تا تعادل نش بازی را تبیین کنیم. فرض بر این است که دو بازیکن همزمان تصمیم گیری می‌کنند.
در پایان بحث با تغییر متغیر های $\gamma $ و $\beta $ تحلیل حساسیت می‌کنیم.
این روند در همه حالات تعادل نش انجام می‌دهیم.

ابتدا بررسی می‌کنیم که $ D_{22} > D_{32} > D_{12} $ و ${ D_{11} > D_{21} > D_{31 $ بنابراین انتخاب سوم پزشک مبنی بر \lr{NCd} قویاً مسلط شده است و در تعادل نش ما نقشی ندارد.

به شکل مشابه داریم ${ P_{31} > P_{32}$  و $ P_{21} > P_{22} $ ،$ P_{11} > P_{12 $.
بنابراین بیمار هم همیشه همکاری را ترجیح میدهد.

در تعادل نش محض ما باید دو عملی که بهترین پاسخ به همدیگر هستند را پیدا کنیم پس بازی با وضعیت مذکور به وضوح تنها یک تعادل نش دارد و آن هم عمل \lr{(C, C)} است.\\

\large \textbf{تحلیل حساسیت}
\normalsize

برای تحلیل حساسیت ما باید بازه هایی که تغییر $\gamma$ یا $\beta$ منجر به تغییر ارجحیت ها می‌شود را پیدا کنیم. از آنجا که $\gamma$ تنها در بیمار موثر است بررسی ستون ها کافیست:\\

1. پزشک همکاری را انتخاب کند
\[
P_{11} = U_1 + B, \quad P_{12} = U_1 - H + \gamma
\]
\[
P_{12} > P_{11} \Rightarrow \gamma > B + H = \gamma_{c}
\]
\\
2. پزشک عدم همکاری موقت را انتخاب کند
\[
P_{21} = U_2 - F_P^t \cdot (U_1 - U_2) + B,& \quad P_{22} = U_2 - R_P^t \cdot (U_1 - U_2) - H + \gamma
\]
\[
P_{22} > P_{21} \Rightarrow \gamma > & B + H + (U_1 - U_2) \cdot (R_P^t - F_P^t) = \gamma_{NCt}
\]
\\
3. پزشک عدم همکاری قطعی را انتخاب کند
\[
P_{31} = U_3 - F_P^d \cdot (U_1 - U_3) + B, \quad P_{32} = U_3 - R_P^d \cdot (U_1 - U_3) - H + \gamma
\]
\[
P_{32} > P_{31} \Rightarrow \gamma > B + H + (U_1 - U_3) \cdot (R_P^d - F_P^d) = \gamma_{NCd}
\]
\\
با توجه به اینکه $ U_2 > U_3 $, $R^d - F^d > R^t - F^t $ میتوان به راحتی دید که $ \gamma_{NCd} > \gamma_{NCt} > \gamma_{C} $ .\\

همچنین چون $\beta $ تنها در ارجحیت های پزشک تاثیرگذار است کافیست تصمیم بیمار را فیکس کنیم و بازه های تاثیر گذاری$ \beta $ را مشخص کنیم. ترتیب حال حاضر ما اگر بیمار همکاری کند$ D_{11} > D_{21} > D_{31}$ است و اگر بیمار همکاری نکند ${ D_{22} > D_{32} > D_{12 $ از آنجا که $\beta $ تنها در ${ D_{31}, D_{32 $ تاثیر گذار است کافیست بتایی را پیدا کنیم که نامساوی های زیر را برقرار کند.
\[
D_{32} = V_3 - F_D^d \cdot (V_1 - V_3) + \beta, \quad D_{22} = V_2 - F_D^t \cdot (V_1 - V_2)
\]
\[
D_{32} > D_{22} \Rightarrow \beta > (V_2 - V_3) + F_D^d \cdot (V_1 - V_3) - F_D^t \cdot (V_1 - V_2) = \beta_{NC}
\]
\[
D_{31} = V_3 - R_D^d \cdot (V_1 - V_3) + \beta, \quad D_{21} = V_2 - R_D^t \cdot (V_1 - V_2)‌
\]
\[
D_{31} > D_{21} \Rightarrow \beta > (V_2 - V_3) + R_D^d \cdot (V_1 - V_3) - R_D^t \cdot (V_1 - V_2) = \beta_{C}
\]
\[
D_{31} = V_3 − R_D^d \cdot (V_1 - V_3) + \beta, \quad D_{11} = V_1
\]
\[
D_{31} > D_{11} \Rightarrow \beta > (V_1 - V_3) \cdot (1 + R_D^d) = \beta_{CC}
\]

باز هم مشابه قبل میتوان به نامساوی زیر رسید
\[
\beta_{CC} > \beta_{C} > \beta_{NC}
\]

برای $\gamma < \gamma_{C} < \gamma_{NCt} < \gamma_{NCd} $ حل مسئله مشابه حالت اولیه است و \lr{(C,C)} تنها تعادل نش بازی است.
این حالت برای تغییرات بتا هم برقرار است مگر $ \beta > \beta_{CC} $ باشد. بنابراین
\[
\gamma < \gamma_{C} \wedge \beta < \beta_{NC} \Rightarrow \{(C,C)\}
\]

اگر در حالت قبلی$ \beta > \beta_{CC}$ باشد داریم $D_{31} > D_{31} $ این مسئله در دنیای واقعی میتواند به دلیل محدودیت های خاص پزشکی مثل انواع سرطان کبد همچون \lr{Hepatocellular Carcinoma} باشد. در این حالت نتیجه زیر برقرار است
\[
\gamma < \gamma_{C} \wedge \beta > \beta_{NC} \Rightarrow \{(NCd,C)\}
\]

فرض کنید $\gamma$ را کمی افزایش دادیم تا برسیم به حالت $\gamma_{C} < \gamma < \gamma_{NCt}  < \gamma_{NCd}   $در این حالت بیمار وقتی پزشک همکاری کند به دلیل لذت الکل ترجیح می‌دهد تا همکاری نکند بنابراین تعادل نش قبلی \lr{(C,C)} برقرار نیست. تنها حالت تعادل نش در این حالت در صورت بالا بودن بتا فرم زیر است
\[
\gamma_{C}  < \gamma < \gamma_{NCt} \wedge \beta > \beta_{NC} \Rightarrow \{(NCd,C)\}
\]

اگر $ \gamma_{C} < \gamma_{NCt} < \gamma < \gamma_{NCd} $ در این صورت بیمار اگر پزشک موقتا همکاری نکند همکاری نمی‌کند.
برای اینکه پزشک هم در این حالت همکاری نکند$ D_{22} > D_{32} $ باید داشته باشیم که $ \beta < \beta_{NC} $ تعادل نش\\ \lr{(NCt,NC) } را خواهیم داشت.
مشابه قبل برای $\beta > \beta_{CC} $  تعادل نش \lr{(NCd,NC) } را داریم.

برای حالت آخر اگر گاما بیشترین مقدار را داشته باشد $ \gamma > \gamma_{NCd} $ یعنی لذت الکل آنقدر زیاد است که بیمار هیچ وقت همکاری نمیکند بنابراین برای $ \beta < \beta_{NC} $  داریم  \lr{(NCt,NC) } و برای $ \beta > \beta_{NC} $  داریم که هر دو طرف عدم همکاری را ترجیح می‌دهند پس \lr{(NCd,NC) } در این حالت یک تعادل نش است.

\subsection{تعادل های نش مخلوط}

مشابه قبل ابتدا $ \beta $ را در کمترین حالت در نظر می‌گیریم و پزشک را بررسی می‌کنیم.
استراتژی \lr{NCd} قویاً مسلط شده است زیرا $E_D[NCd] $ از $ E_D[C] = E_D[NCt] $ کوچکتر است.
اگر فرض کنیم احتمال همکاری بیمار $p$ و احتمال عدم همکاری او $ (1-p) $ در نظر بگیریم(دقت کنید که فقط دو استراتژی داریم.) پس به معادله زیر می‌رسیم.

\[
p = \frac{G - F_D^t - 1}{G - F_D^t + R_D^t}
\]

به طور مشابه برای بیمار اگر فرض کنیم که پزشک با احتمال $q$ همکاری و با احتمال $(1-q) $ عدم همکاری را انتخاب می‌کند به رابطه زیر می‌رسیم.
\[
q = 1 - \frac{\gamma - (B + H)}{(R_P^t - F_P^t) \cdot (U_1 - U_2)}
\]

اگر مقدار $\gamma$ خیلی کم باشد بیمار همیشه همکاری را انتخاب می‌کند و تعادل نش مخلوط نداریم اگر مقدار $\gamma$ هم خیلی زیاد باشد بیمار همیشه عدم همکاری را انتخاب می‌کند و باز تعادل نش مخلوط نداریم پس نتیجه می‌گیریم که $\gamma_{C} < \gamma < \gamma_{NCt} $ حال می‌توانیم روی $\gamma$ و  بر حسب $ p$ و $q$، تحلیل حساسیت کنیم.

اگر $  B + H + (R_D^t+1) \cdot (R_P^t-F_P^t) \cdot (U_1-U_2) \cdot  (G-F_D^t+R_D^t)^{-1} > \gamma > \gamma _{C}  $ ، آنگاه $ q>p $ و دکتر با احتمال بیشتری نسبت به بیمار همکاری می‌کند؛\\ و
اگر$ B + H +(R_D^t+1) \cdot (R_P^t-F_P^t) \cdot (U_1-U_2) \cdot  (G-F_D^t+R_D^t)^{-1} < \gamma < \gamma_{NCt} $ ، آنگاه $p>q $ پس بیمار با احتمال بیشتری همکاری میکند.
این شاید غیرشهودی به نظر برسد که در صورت لذت بیشتر نوشیدن الکل احتمال همکاری بیمار بیشتر شود ولی باید دقت داشته باشید که در تعادل نش مخلوط هدف ما بی تفاوتی رقیب بین انتخاب های ماست، بنابراین طبیعی است که اگر بیمار میل بیشتری به عدم همکاری داشته باشد میزان همکاری او افزایش یابد.\\

احتمال همکاری بیمار $q$ به طور مستقیم با \lr{B} و \lr{H} رابطه دارد یعنی در صورت افزایش این دو پارامتر احتمال همکاری بیمار افزایش می‌یابد که باعث می‌شود دکتر هم با احتمال بیشتری همکاری کند. با اینحال چون ما در حال بررسی  تعادل های مخلوط هستیم پزشک در جهت عکس $p$ حرکت می‌کند پس افزایش پارامتر های \lr{B} و \lr{H} باعث افزایش همکاری بیمار و کاهش همکاری پزشک می شود و کاهش آنها نتیجه عکس دارد.

\section{نتیجه گیری}

در تعادل نش مطلق دیدیم که در صورت پایین بودن مقادیر $\beta$ و $\gamma$ بهترین گزینه مشترک همکاری طرفین است.
در تحلیل حساسیت دیدیم که پایین بودن مقدار $\beta$ در حالات مختلف نتیجه مطلوبی ندارد و باعث سواستفاده بیمار از دکتر می‌شود، زیاد بودن این متغیر هم تاثیر منفی دارد و باعث می‌شود دکتر به کسانی که لایق هم هستند کبد ندهد، همچنین دیدیم مقدار $\gamma$ هرچه کمتر باشد بهتر است. پس بهترین گزینه منطقی برای تعادل مطلق مقدار متعادل $\beta$ و کمینه بودن $\gamma$  است.\\

در بررسی تعادل نش مخلوط دیدیم که برای که برای اکثر حالات متغیرات توازن احتمالی خوبی داریم، برای مقادیر پایین $\gamma$  دکتر بیشتر همکاری می‌کند و برای مقادیر بالای $\gamma$  بیمار بیشتر همکاری می‌کند.
بررسی مقایسه ای احتمالی نشان داد که $q$ یعنی احتمال همکاری دکتر به طور مستقیم با $\gamma$ و $B$ و $H$ رابطه دارد و افزایش این متغیر ها به افزایش احتمال همکاری دکتر می‌انجامد.
نتیجه گیری تحلیلی این موضوع به ما نشان می‌دهد که همکاری بیشتر دکتر در صورت همکاری بیشتر بیمار است، همچنین آگاه کردن بیمار از ضررات مصرف الکل و فواید ترک آن در احتمال همکاری بیمار موثر است. جدای این مسائل میزان همکاری بیمار تنها وابسته به احساسات دکتر است برای مثال اگر دکتر احساس گناه زیادی بکند یا استیصال زیادی را تجربه کند بیمار باید بیشتر برای ماندن در لیست همکاری کند.\\

\section*{قدردانی}
از استادان گرانقدر دکتر مهدی رضا درویش زاده و دکتر مرتضی محمدنوری که ما را در  تبیین و ثبت و ارائه مسئله یاری رساندند، صمیمانه سپاسگزاری میکنم.

\setLTRbibitems
\begin{thebibliography}{99}
	\resetlatinfont
	\bibitem{A}
	Carrion, A. F., L. Aye, and P. Martin 2013. Patient
	selection for liver transplantation. \textit{Expert Review
		of Gastroenterology \& Hepatology}, 7(6):571–579.
	\bibitem{A}
	Diamond, G. A., A. Rozanski, and M. Steuer 1986.
	Playing doctor: Application of game theory to
	medical decision-making. \textit{Journal of Chronic Diseases}, 39(9):669–677.
	\bibitem{A}
	Djulbegovic, B., I. Hozo, and J. P. Ioannidis 2015.
	Modern health care as a game theory problem. \textit{European Journal of Clinical Investigation},
	45(1):1–12.
	\bibitem{A}
	Elwyn, G. 2004. The consultation game. \textit{Quality
		safety in health care}, 13(6):415–416.
	\bibitem{A}
	Henry, M. S. 2006. Uncertainty, responsibility, and
	the evolution of the physician/patient relationship. \textit{Journal of Medical Ethics}, 32(6):321–323.
	\bibitem{A}
	Kock, M. 2004. Disability law in germany: An
	overview of employment, education and access
	rights. \textit{German Law Journal}, 5(11):1373–1392.
	\bibitem{A}
	Marinho, R. T., H. Duarte, J. Gíria, J. Nunes,
	A. Ferreira, and J. Velosa 2015. The burden
	of alcoholism in fifteen years of cirrhosis hospital admissions in portugal. \textit{Liver International},
	35(3):746–755.
	\bibitem{A}
	Telles-Correia, D. and I. Mega 2015. Candidates for
	liver transplantation with alcoholic liver disease:
	Psychosocial aspects. \textit{World Journal of Gastroennal of Gastroenteterology}, 21(39):11027–11033.
\end{thebibliography}
\section{پیوست}
\subsection{بازی}
یک بازی شامل مجموعه‌ای از بازیکنان، مجموعه‌ای از حرکت‌ها یا راه بردها و نتیجه مشخصی برای هر ترکیب از راه بردها می‌باشد. پیروزی در هر بازی تنها تابع شانس نیست بلکه اصول و قوانینِ ویژه خود را دارد و هر بازیکن در طی بازی سعی می‌کند با به‌کارگیری آن اصول، خود را به بُرد نزدیک کند.

\subsection*{ساختار بازی}
هر بازی از سه بخشِ اساسی زیر تشکیل شده‌است:

1. \textbf{بازیکن‌ها}.
بازیکنان، تصمیم گیرندگان بازی می‌باشند. بازیکن می‌تواند شخص، شرکت، دولت و $\dots$  باشد.

2. \textbf{کُنش‌ها}.
مجموعه‌ای است از تصمیمات و اقداماتی که هر بازیکن می‌تواند انجام دهد.


3. \textbf{نمایه عمل}.
هر زیر مجموعه‌ای از مجموعه اعمال ممکن را یک نمایه گوییم.

\subsection{تابع سود}
اولویت‌های یک بازیکن در اصل مشوق‌های بازیکن برای گرفتن یا نگرفتن تصمیمی می‌باشد؛ و به عبارت دیگر بیان گر نتیجه و امتیاز بازیکن در صورت گرفتن تصمیم متناظر با آن می‌باشد.
\subsection{استراتژی}
در استراتژی (نظریه بازی)\LTRfootnote{\lr{Strategy(game theory)}} استراتژی یا راهبرد یک بازیکن در یک بازی یک مجموعه کامل از اعمالی است که در هر موقعیت انجام می‌دهد.
استراتژی به‌طور کامل رفتار بازیکن را بیان می‌کند. استراتژی یک بازیکن بیان‌کننده اعمالی است که بازیکن در هر مرحله از بازی، برای هر مجموعه از اعمالی که بازیکن قبل از این مرحله انجام داده، انتخاب می‌کند.
یک نمایه استراتژی(گاهی آن را ترکیب استراتژی نیز می‌نامند) مجموعه‌ای از استراتژی‌ها برای هر بازیکن است که به‌طور کامل همه اعمال در یک بازی را بیان می‌کند. یک نمایه استراتژی باید شامل یک و فقط یک راهبرد برای هر بازیکن باشد.
مفهوم استراتژی گاهی به غلط با حرکت اشتباه گرفته می‌شود. یک حرکت عملی است که توسط یک بازیکن در نقطه‌ای از بازی انتخاب می‌شود (مثلاً در شطرنج حرکت فیل سفید از نقطه $a2$ به نقطه $b3$.) در حالی که یک استراتژی یک الگوریتم کامل برای انجام بازی است که به بازیکن می‌گوید در هر موقعیت ممکن در طول بازی چه کار کند.

\subsection{نظریه بازی}

نظریه بازی یا نگره بازی با استفاده از مدل‌های ریاضی به تحلیل روش‌های همکاری یا رقابت موجودات منطقی و هوشمند می‌پردازد. نگره بازی، شاخه‌ای از ریاضیات کاربردی است که در علوم اجتماعی و به ویژه در اقتصاد، زیست‌شناسی، مهندسی، علوم سیاسی، روابط بین‌الملل، علوم رایانه، بازاریابی، فلسفه و قمار مورد استفاده قرار می‌گیرد. نگره بازی در تلاش است تا بوسیله ریاضیات، رفتار را در شرایطِ راهبردی یا در یک بازی که در آن‌ها موفقیتِ فرد در انتخاب کردن، وابسته به انتخاب دیگران می‌باشد، برآورد کند.

نگره بازی تلاش می‌کند تا رفتار ریاضی حاکم بر یک موقعیت راهبردی (تضارب منافع) را مدل‌سازی کند. این موقعیت، زمانی پدید می‌آید که موفقیتِ یک فرد وابسته به استراتژی هایی است که دیگران انتخاب می‌کنند. هدفِ نهاییِ این دانش، یافتنِ استراتژی بهینه برای بازیکنان است.

در ابتدا نگره بازی معادل با بازی مجموع-صفر بود، که در آن سود (یا زیان) یک شرکت‌کننده، دقیقاً متعادل با زیان‌های (یا سودهای) سایر شرکت کنندگان می‌باشد و بازیکن‌ها چیزی را به دست می‌آورند که بازیکن دیگری آن را از دست داده باشد.

امروزه نگره بازی یک واژه مادر برای علومی که به تحلیل رفتار منطقی متقابل انسان‌ها، حیوانات و رایانه‌ها می‌پردازند می‌باشد.

\subsection*{رفتار بخردانه یا عقلایی}

اصل مهمِ نظریه بازی‌ها بر بخردانه بودن رفتار بازیکنان است. بخردانه بودن به این معنا است که هر بازیکن تنها در پی بیشینه کردنِ سودِ خود بوده و هر بازیکن می‌داند که چگونه می‌تواند سودِ خود را بیشتر کند؛ بنابراین حدس زدنِ رفتار ایشان که بر اساس نمودار هزینه-فایده است آسان خواهد بود. مانند بازی شطرنج که می‌توان حدس زد که حریف بازیِ با تجربه چه تصمیمی خواهد گرفت.

\subsection{تعادل نش}
تعادل نش\LTRfootnote{\lr{Nash equilibrium}} مفهومی در نظریه بازی‌ها است که کاربرد فراوانی در اقتصاد پیدا کرده و نام آن از جان فوربز نش گرفته شده‌است.
در تئوری بازی‌ها، تعادل نش (به نام جان فوربز نش، که آن را پیشنهاد کرد) راه حلی از نظریه بازی است که شامل دو یا چند بازیکن می‌شود. در این راه حل فرض بر آگاهی هر بازیکن به استراتژی تعادل بازیکنان دیگر است، بدون وجود هیچ بازیکنی که فقط برای کسب سود خودش با تغییر راهبرد یک جانبه عمل کند. اگر هر بازیکنی راهبرد را انتخاب کنند، هیچ بازیکنی نمی‌تواند با تغییر راهبرد خود در حالی که نفع بازیکن دیگر را بدون تغییر نگه داشته باشد عمل کند، سپس مجموعه انتخاب‌های راهبرد فعلی و بهره‌مندی مربوطه، تعادل نش را تشکیل می‌دهد.
به بیان ساده، امی و فیل در تعادل نش است، اگر امی در حال انجام بهترین تصمیم‌گیری که او می‌تواند با توجه به تصمیم‌گیری فیل داشته باشد و همچنین فیل بهترین تصمیمی که می‌تواند با توجه به تصمیم‌گیری امی داشته باشد. به همین ترتیب یک گروه از بازیکنان در تعادل نش است. اگر هر یک در حال انجام بهترین تصمیم‌گیری باشند که آن‌ها می‌توانند با توجه به تصمیمات دیگران داشته باشند، با این حال، تعادلی که نش است لزوماً به معنای بهترین بهره‌وری کل برای همه بازیکنان مربوطه نمی‌باشد زیرا در بسیاری از موارد ممکن است تمام بازیکنان بهره‌وری خود را بهبود بخشند در صورتی که چگونه بتوانند به توافق بر روی استراتژی‌های مختلف از تعادل نش برسند. (به عنوان نمونه، شرکت‌های تجاری رقابتی به منظور افزایش سود آن‌ها تشکیل کارتل می‌دهد).
جنبه مهم تعادل نش این است که سود هر بازیکن نه تنها به استراتژی  برگزیده خود بلکه به استراتژی  برگزیده دیگر بازیکنان نیز ارتباط دارد.\\

\textbf{تعریف رسمی}.
مجموعه \lr{(S,F)} به عنوان بازی با $n$ بازیکن مفروض است، که در آن $S_{i} $ مجموعه استراتژی ها برای بازیکن $i$ است. مجموعه $S = \prod_{\substack{i=1}}^n S_{i} $ مجموعه‌ای از فضای برد آن است و $F(x)$ تابع سود آن است. همچنین $X_{-i}$ به عنوان فضای استراتژی همه بازیکنان به جز بازیکن $i$ مفروض است. هر بازیکن به ازای هر $i \in \Z$، استراتژی $X_{i}$ را انتخاب کند، پروفایل استراتژی آن بصورت ${ x=x_1, x_2, \dots, x_{n$ و تابع سود آن بصورت $F(x_{i})$ خواهد بود. تابع سود به نمای استراتژِی انتخابی وابسته است؛ به عبارت دیگر در استراتژی منتخب بازیکن $i$ و نیز استراتژِی های منتخب دیگر بازیکنان، نمای $x^* \in S$ یک تعادل نش است اگر هیچ انحراف یک سویی در استراتژی توسط هر بازیکن واحد با یکی دیگر از بازیکنان، سودآور نباشد. یعنی:
\[
 \forall i \in [n], x_{i}\in S_{i}, x_{i}\neq x_{i}^{*}(f_{i}(x_{i}^{*}, x_{-i}^{*})\geq f_{i}(x_{i}, x_{-i}^{*}))
\]

یک بازی می‌تواند یا راهبرد محض یا تعادل نش ترکیبی باشد،(در تعریف اخیر استراتژی محض آن است که به صورت تصادفی با فراوانی ثابت انتخاب شده‌است). نش نشان داد که در صورتی به ما اجازه استراتژی ترکیب شده بدهند، سپس هر بازی با تعداد محدودی از بازیکنان که در آن هر بازیکن می‌تواند به صورت غیر محدود از میان بسیاری از راهبرد‌های کامل که حداقل یک تعادل نش می‌باشد انتخاب کند. وقتی نابرابری اکید در بالا نگه می‌دارد برای تمام بازیکنان و تمام راهبرد‌های جایگزین امکان‌پذیر است، سپس تعادل طبقه‌بندی شده به عنوان یک تعادل دقیق نش می‌باشد. اگر در عوض، برای برخی از بازیکنان، برابری دقیقی بین $x$ و بعضی از استراتژی‌های مجموعه $S$ وجود دارد. سپس تعادل به عنوان یک تعادل طبقه‌بندی شده ضعیفی از نش می‌باشد.
اگر بازی دارای یک تعادل نش منحصربه‌فرد است و در میان بازیکنان تحت شرایط خاصی انجام می‌شود، بنابراین مجموعه راهبرد تعادل نش تصویب شده خواهد بود. شرایط کافی برای تضمین اینکه تعادل نش در حال بازی شدن است عبارت اند از:

بازیکنان تمام قدرت خود برای حداکثر کردن بهره‌وری مورد انتظار خود در بازی‌های مورد انتظار انجام می‌دهند؛ بی نقص در اجرای بازی هستند؛ هوش کافی برای استنتاج راه حل دارند؛ از راهبرد تعادلی برنامه‌ریزی شده توسط بازیکن‌های دیگر با خبرند و بازیکنان بر این باورند که انحراف در راهبرد خود باعث انحراف هر بازیکن دیگر نخواهد شد؛
شناخت مشترک که همه بازیکنان با این شرایط مواجه‌اند وجود دارد.

جان فوربز نش، در مقاله معروف خود اثبات کرد که برای هر بازی متناهی یک تعادل وجود دارد. تعادل نش بر دو قسم است: تعادل‌های نش راهبرد خالص تعادل‌هایی هستند که در آن‌ها همه بازیکنان با راهبرد خالص بازی می‌کنند. تعادل‌های نش راهبرد مختلط تعادل‌هایی هستند که در آن‌ها حداقل یک بازیکن با راهبرد مختلط بازی می‌کند. نش اثبات کرد که هر بازی متناهی یک تعادل نش دارد، نه این که هر بازی متناهی یک تعادل نش خالص دارد. بازی‌ها می‌توانند هم تعادل خالص و هم تعادل مختلط داشته باشند.
\end{document}
